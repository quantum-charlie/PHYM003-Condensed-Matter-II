%
% Report Template Version 1.0
% University of Exeter
% Department of Physics and Astronomy
%
%
% Comments start with % (percent) character and last till the end of the line.
%
% Compile this latex file using pdflatex command, rather than the latex 
% command, to produce pdf output directly  
%
% LaTeX2e document starts with \documentclass[options]{<class-name>}
%
% <class-name> can be one of the standard LaTeX document classes: 
% article, report or book, or some other specialised class.
%
% options: a4paper - paper size; onecolumn or two column format; 12pt font size
%
% always use a minimum of a 12pt font throughout for readability 
%
% uncomment/comment the appropriate documentclass of 2 options below
%
%\documentclass[a4paper,onecolumn,12pt]{article}
\documentclass[a4paper,twocolumn,11pt]{article}
%
% Preamble of LaTeX document is everything before \begin{document}.
%
% Preamble is used to load extension packages and to set up global 
% parameters and configuration for the entire document.
%
% Extension packages providing additional functionality:
%

\usepackage{amsmath}      % additional math environments
\usepackage{graphicx}     % graphics import from external files 
\usepackage{caption}      % customisation of captions
\usepackage{booktabs}     % table typesetting additions
\usepackage{url}          % format url addresses
\usepackage{abstract}	  % allows formatting of abstract
\usepackage{wasysym}      % provides astronomical symbols
\usepackage{txfonts}	     % nicer roman font than the default
\usepackage{comment}	     % allows one to use \begin{comment} and \end{comment} to comment out section

\usepackage{braket}       % added 28/01/20
\usepackage{dsfont}       % added 18/12/20
\usepackage{rotating}     % added 22/12/20
\usepackage{supertabular} % added 22/12/20
\usepackage{array}        % added 22/12/20
\newcolumntype{P}[1]{>{\centering\arraybackslash}p{#1}}
\makeatletter
\renewcommand{\maketag@@@}[1]{\hbox{\m@th\normalsize\normalfont#1}}
\makeatother
\DeclareMathOperator{\sech}{sech}
\DeclareMathOperator{\csch}{csch}
\DeclareMathOperator{\arcsec}{arcsec}
\DeclareMathOperator{\arccot}{arcCot}
\DeclareMathOperator{\arccsc}{arcCsc}
\DeclareMathOperator{\arccosh}{arcCosh}
\DeclareMathOperator{\arcsinh}{arcsinh}
\DeclareMathOperator{\arctanh}{arctanh}
\DeclareMathOperator{\arcsech}{arcsech}
\DeclareMathOperator{\arccsch}{arcCsch}
\DeclareMathOperator{\arccoth}{arcCoth} 
%
% OPTIONS FOR REFERENCES
%
% If you reference by numbers (e.g. [1]), you can enter references manually, or create a BibTeX file that 
% contains all your references.  This is particularly useful if you know you will be using references again 
% and again, or if you want to keep a library of references all in one place.
%
% This line defines a logical variable `usebibtex' that specifies that you do (true) or do not want (false) to
% use BibTeX.  The name of the BibTeX file is set near the end of the document using the 
% \bibliography{bibliography} command.
%
\newif\ifUseBibTeX
%
% These two lines define the logical variable to be either true or false: comment/uncomment the one you want:
%
\UseBibTeXtrue
%\UseBibTeXfalse
%
%
% Many physics articles use numbers for references (e.g. [1])
% Most astronomy and astrophysics articles refer to other articles by Author Name and Year. (e.g. Jones 2010)
% Choose which to use below:
%
% This line defines a logical variable `RefByNum'
\newif\ifRefByNum
%
% These two lines define the logical variable to be either true or false: comment/uncomment the one you want:
%
\RefByNumtrue
%\RefByNumfalse
%
% The following lines, down to \fi (which is the end of the if section), are required depending on which type of
% referencing you choose
%
\ifRefByNum
%
% For standard reference by number, use:
%
\usepackage{cite}         % improved handling of numeric citations
\bibliographystyle{ieeetr}   % for this style the numbers are assigned in the order they are referenced in the text
%
%
\else
%
% For reference by author and year, use:  
%
\UseBibTeXfalse
\usepackage{aas_macros}
\usepackage[round]{natbib}
\setcitestyle{aysep={}}
\bibliographystyle{mn2e}
%
%      You also need to have the following files in the directory:
%         aas_macros.sty    - this contains abbreviations of many journal names
%         mn2e.bst              - this allows you to use a bibtex file to contain the full reference information
%
%      You will need to create a ".bib" file containing bibtex records for each reference.
%	For example, if you use the NASA ADS system to find papers, it can give you the bibtex entry to copy 
%	and paste into your ".bib" file.  See, for example, http://adsabs.harvard.edu/abs/2018MNRAS.475.5618B
%
%	Which gives a bibtex entry like:
%	@ARTICLE{2018MNRAS.475.5618B,
%	   author = {{Bate}, M.~R.},
%	    title = "{On the diversity and statistical properties of protostellar discs}",
%	  journal = {\mnras},
%	archivePrefix = "arXiv",
%	   eprint = {1801.07721},
%	 primaryClass = "astro-ph.SR",
%	 keywords = {accretion, accretion discs, hydrodynamics, radiative transfer, methods: numerical, protoplanetary discs, stars: formation},
%	     year = 2018,
%	    month = apr,
%	   volume = 475,
%	    pages = {5618-5658},
%	      doi = {10.1093/mnras/sty169},
%	   adsurl = {http://adsabs.harvard.edu/abs/2018MNRAS.475.5618B},
%	  adsnote = {Provided by the SAO/NASA Astrophysics Data System}
%	}
%
%. At the end of your LaTeX document, you then reference the bibliography using:
% \bibliography{bibtexfilename}
%     where your bib file has, for example, the name: bibtexfilename.bib
%
\fi
%
% END OF DEFINITIONS FOR REFERENCES
%
% PAGE FORMATTING
%
% To set margins etc.:
%
\textheight 24.0cm        % sets the length of the text on each page
\textwidth 16.0cm         % sets the width of the text on each page
\topmargin -1.25cm        % sets the top margin: - higher, + lower 
\oddsidemargin 0.5cm      % makes the left margin: + wider, - narrower
%
% \renewcommand{\abstractname}{}    % removes abstract title
%
% sets abstract margins, but no real need to do this:
% \setlength{\absleftindent}{30mm}
% \setlength{\absrightindent}{30mm}
%
% some handy commands for referencing;
% the optional argument overrides the default label, e.g.
% \figref[FIG.~]{fig:label}
\newcommand{\figref}[2][\figurename~]{#1\ref{#2}}
\newcommand{\tabref}[2][\tablename~]{#1\ref{#2}}
\newcommand{\secref}[2][Section~]{#1\ref{#2}}
%
%
% BEGINNING OF THE ACTUAL DOCUMENT
%
% The document opens with \begin{document} and closes with \end{document}
%
\begin{document}
%
% The aim of the formatting should be to optimise readability, hence reports
% should be single column and double line spaced, with an appropriate number of 
% words per line - between 8 and 12. 
% edit title and author as needed:
%
\title{\textbf{\Huge PHYM003 - Condensed Matter II}}           % fill in the title here
\author{\LARGE Key Equations and Concepts}         % fill in your name here
%
\date{\Large \textit{Academic Year 20/21}}  % resets date of the report from today's date
%
%\twocolumn[	          % makes title and abstract appear over entire page 
                          % width - possibly required if two-column option 
                          % chosen in documentclass - otherwise comment out
%
\maketitle                % formats the title
%
\section{Drude Theory in Metals (Part 1)}

Drude theory is a kinetic theory of gases applied to a metal, considered as a gas of electrons.

\subsection{Assumptions}

\textit{From lecture slides:}
\begin{enumerate}
    \item{electrons are solid spheres moving in straight lines until a collision event takes place}
    \item{time taken up by a collision is negligible}
    \item{no force acts on the particles between collisions}
    \item{electrons achieve thermal equilibrium through collisions}
\end{enumerate}

\subsection{Free Electron Density}
\begin{equation}
    n = \frac{N}{\mathcal{V}} = \frac{N_A\mathcal{Z}\rho_m}{A},
\end{equation}
where $\mathcal{Z}$ is the atomic number, $\rho$ is the molar mass, and $A$ is the atomic mass.

\subsection{Molar Density}
\begin{equation}
    \rho_\text{molar} = \frac{\rho_\text{mass}}{A},
\end{equation}
where $A$ is the atomic mass.

\subsection{Electric Field}
\begin{equation}
    \mathbf{E} = \rho\mathbf{j},
\end{equation}
where $\rho$ is the resistivity and $\mathbf{j}$ is the current density. This is essentially a restatement of Ohm's law.

\subsection{Current Density}
\begin{equation}
    \mathbf{j} = -ne\mathbf{v} = \sigma\mathbf{E}
\end{equation}
where $n$ is the electron number density, $e$ is the electron charge, $\mathbf{v}$ is the electron velocity, and $\sigma$ is the electrical conductivity.

\subsection{Average Additional Velocity Due to Electric Field}
\begin{equation}
    \mathbf{v}_\text{av} = -\frac{e\mathbf{E}\tau}{m},
\end{equation}
where $\tau$ is the mean free time, and $m$ is the electron mass.

\subsection{DC Conductivity}
\begin{equation}
    \sigma_0 = \frac{ne^2\tau}{m} = ne\mu
\end{equation}

\subsection{Mobility}
\begin{equation}
    \mu = \frac{e\tau}{m}
\end{equation}
Mobility has units of $m^2V^{-1}s^{-1}$.

\subsection{Mean Free Path}
\begin{equation}
    \ell = \mathbf{v}_0\tau
\end{equation}

\subsection{Lorentz Force}
\begin{equation}
    \mathbf{F} = -e(\mathbf{E} + \mathbf{v}\times\mathbf{B}),
\end{equation}
where $B$ is the magnetic field.

\subsubsection{Momentum EoM}
\begin{equation}
\begin{split}
    \mathbf{p}(t+dt) &= \bigg(1-\frac{dt}{\tau}\bigg)[\mathbf{p}(t) + \mathbf{F}(t)dt]\\
    &= \mathbf{p}(t) - \bigg(\frac{dt}{\tau}\bigg)\mathbf{p}(t) + \mathbf{F}(t)dt - \frac{\mathbf{F}(t)}{\tau}(dt)^2.
\end{split}
\end{equation}
Considering only terms linear in $dt$:
\begin{equation}
    \mathbf{p}(t+dt) = \mathbf{p}(t) - \bigg(\frac{dt}{\tau}\bigg)\mathbf{p}(t) + \mathbf{F}(t)dt
\end{equation}
\begin{equation}
    \therefore \frac{d\mathbf{p}(t)}{dt} = - \frac{\mathbf{p}(t)}{\tau} + \mathbf{F}(t)
\end{equation}

\subsection{The Hall Effect}

Presence of a magnetic field results in the build up of charge carriers on the edges of the material. This produces a voltage and therefore a resistance across the conductor.

\subsubsection{Magnetic Field}
\begin{equation}
    \mathbf{B} = \mu_0(\mathbf{H} + \mathbf{M}),
\end{equation}
and if the material is not ferromagnetic $\mathbf{B}\approx\mu_0\mathbf{M}$.

\subsubsection{Longitudinal Magnetoresistance}
\begin{equation}
    \rho_{xx}(B) = \frac{E_x}{j_x}
\end{equation}
where $B$ is the magnetic field.

\begin{figure}[h!]
\centering 
\includegraphics[width=70mm]{images/hall_effect.png}
\end{figure}

\subsection{Hall Voltage}
\begin{equation}
    V_H = \frac{I_x B_z}{nte}
\end{equation}

\subsection{Hall Coefficient}
\begin{equation}
    R_H = \rho_{xy}(B) = \frac{E_y}{j_xB_z} = -\frac{1}{ne}
\end{equation}

\section{Drude Theory in Metals (Part 2)}

\subsection{AC Conductivity}
\begin{equation}
    \sigma(\omega) = \frac{\sigma_0}{1-i\omega\tau}
\end{equation}

\subsection{Complex Dielectric Constant}
\begin{equation}
    \epsilon(\omega) = 1 + \frac{4\pi i\sigma}{\omega}
\end{equation}
At high frequencies ($\omega\tau >> 1$), this becomes:
\begin{equation}
    \epsilon(\omega) = 1 - \frac{\omega_p^2}{\omega^2},
\end{equation}
where $\omega_p$ is the plasma frequency.

\subsection{Plasma Frequency}
\begin{equation}
    \omega_p^2 = \frac{4\pi ne^2}{m}
\end{equation}

\section{Ground-State Properties of the Electron Gas}

\subsection{Maxwell-Boltzmann Distribution}
\begin{equation}
    f(v) = \bigg(\frac{m}{2\pi k_BT}\bigg)^{3/2}\exp{\bigg(-\frac{mv^2}{2k_BT}\bigg)}
\end{equation}

\subsection{Fermi-Dirac Distribution}

\begin{equation}
    f(v) = \frac{(m/\hbar)^3}{4\pi^3}\frac{1}{\exp{\bigg(\frac{\frac{1}{2}mv^2-k_BT_0}{k_BT}+1\bigg)}}
\end{equation}

\subsection{Schrödinger Equation}

\begin{equation}
    -\frac{\hbar^2}{2m}\boldsymbol{\nabla}^2\psi^2(\mathbf{r}) = E\psi(\mathbf{r})
\end{equation}

\subsection{Number of Allowed Values of $k$ in Region $\Omega$ of $k$ Space}

\begin{equation}
    N_k = \frac{\Omega}{\big(\frac{2\pi}{L}\big)^3} = \frac{\Omega\mathcal{V}}{8\pi^3}
\end{equation}

\subsection{Number of Electrons in k Space}

\subsubsection{3D}
\begin{equation}
    N = g_s\bigg(\frac{4}{3}\pi k_F^3\bigg)\bigg(\frac{a}{2\pi}\bigg)^3
\end{equation}
where $g_s$ is the spin degeneracy, and $k_F$ is the Fermi wavevector.

\subsubsection{2D}
\begin{equation}
    N = g_s\bigg(\pi k_F^2\bigg)\bigg(\frac{a}{2\pi}\bigg)^2
\end{equation}

\subsubsection{1D}
\begin{equation}
    N = g_s\big(2k_F\big)\bigg(\frac{a}{2\pi}\bigg)
\end{equation}

\subsection{Density of States}

\subsubsection{Function of k}
\begin{equation}
    \rho(k) = \frac{1}{a^D}\frac{dN}{dk},
\end{equation}
where $D$ is the number of dimensions.

\subsection{Function of E}
\begin{equation}
    \rho(E) = \rho(k)\frac{dk}{dE}
\end{equation}

\section{Geometrical Description of Crystals and Band Theory}

\subsection{Translation Vector}

\begin{equation}
    \mathbf{R} = n_1\mathbf{a}_1 + n_2\mathbf{a}_2 + n_3\mathbf{a}_3
\end{equation}
where $\mathbf{a}_n$ are the primitive or fundamental translational vectors.

\subsection{Unit Cell Volume}

\begin{equation}
    \Omega = \mathbf{a}_1 \cdot (\mathbf{a}_2 \times \mathbf{a}_3),
\end{equation}
where $\mathbf{a}_n$ are the primitive or fundamental translational vectors.

\subsection{Bloch's Theorem}
\begin{equation}
    \psi_{n\mathbf{k}}(\mathbf{r}) = e^{i\mathbf{k \cdot r}}U_{n\mathbf{k}}(\mathbf{r}),
\end{equation}
where $n$ is the energy-band index. Bloch's theorem can also be expressed as:
\begin{equation}
    \psi_{n\mathbf{k}}(\mathbf{r+R}) = \psi_{n\mathbf{k}}(\mathbf{r})e^{i\mathbf{k \cdot R}}.
\end{equation}

\subsection{Tight Binding}

\subsubsection{Schrödinger Equation}
\begin{equation}
    H\psi = \bigg(-\frac{\hbar^2}{2m}\boldsymbol{\nabla}^2 + U(\mathbf{r})\bigg)\psi = E\psi,
\end{equation}
where $U(\mathbf{r})$ is the periodic potential $U(\mathbf{r}) = U(\mathbf{r+R})$.

\subsubsection{Bloch Sum (LCAO) Method}
\begin{equation}
    \Phi(k,x) = \frac{1}{\sqrt{N}}\sum_n e^{ikt_n}\varphi_n(x-t_n),
\end{equation}
where $N$ is the number of unit cells for the crystal, and $\varphi_n$ represents the atomic orbital.

\subsubsection{General Expression for Energy}
\begin{equation}
    E = \braket{\Phi|\hat{H}|\Phi}
\end{equation}

\subsubsection{Onsite Energy}
\begin{equation}
    E_0 = \braket{\varphi_n(x-t_n)|\hat{H}|\varphi_n(x-t_n)},
\end{equation}
where the nearest neighbour translation vectors are given by $t_n = na$.

\subsubsection{Nearest-Neighbour Energy}
\begin{equation}
    \gamma = \braket{\varphi_n(x-t_n)|\hat{H}|\varphi_n(x-t_{n\pm1})}
\end{equation}

\subsubsection{1D Square Lattice Energy}
\begin{equation}
    E = E_0 + 2\gamma\cos(k_xa)
\end{equation}

\subsubsection{2D Square Lattice Energy}
\begin{equation}
    E = E_0 + 2\gamma\big[(\cos(k_xa) + \sin(k_ya)\big]
\end{equation}

\subsubsection{3D Square Lattice Energy}

\small
\begin{equation}
    E = E_0 + 2\gamma\big[(\cos(k_xa) + \sin(k_ya) + + \sin(k_za)\big]
\end{equation}
\normalsize

\subsubsection{Graphene Energy}

\footnotesize
\begin{equation}
    E = \pm \beta \sqrt{1+4\cos{\bigg(\frac{\sqrt{3}}{2}ak_y\bigg)}\cos{\bigg(\frac{a}{2}ak_x\bigg)}+4\cos^2{\bigg(\frac{a}{2}k_x\bigg)}}
\end{equation}
\normalsize
\noindent
Tight binding works well for many systems but not metal dichalcogenides, for example.

\subsubsection{Effective Mass}
\begin{equation}
    m^* = \frac{\hbar^2}{2\gamma a^2}
\end{equation}

\section{Band Theory of Crystals}

\subsection{One-Electron Schrödinger Equation}
\begin{equation}
    \bigg(\frac{p^2}{2m} + V(\mathbf{r})\bigg)\psi(\mathbf{r}) = E\psi(\mathbf{r}),
\end{equation}
where $V(\mathbf{r+t_n}) = V(\mathbf{r})$.

\subsection{Approximations}

\textit{From lecture slides:}
\begin{enumerate}
    \item{\textbf{rigid lattice approximation}: the nuclei are taken as fixed at their equilibrium positions; the large difference between the masses of the electrons and the nuclei is the basic justification for this}
    \item{\textbf{one-electron approximation in local form}: in essence, the complicated many-body electron problem is simplified to a single one-electron problem with an appropriate local potential}
    \item{\textbf{relativistic effects are neglected}: whenever necessary, one should replace the Schrödinger equation with the Dirac equation. Often one includes relativistic terms of interest (such as spin-orbit coupling) by perturbation theory}
\end{enumerate}

\subsection{Single-Cell Formulation}

In the cellular method attention is focused to the Wigner-Seitz unit cell. Within it, the crystal potential is assumed to be spherically symmetric. The radial Schrodinger equation is solved, and appropriate boundary conditions at the surface of the unit cell are applied. In many practical cases the assumption of a spherically symmetric crystal potential is well justified because of the dominant contribution of the atomic potential at the centre of the Wigner-Seitz cell.

\subsubsection{Wavefunction}

\begin{equation}
    \psi(\mathbf{k,r}) = \sum_{\ell, m}c_{\ell, m}(\mathbf{k})Y_{\ell, m}(\mathbf{r})R_\ell(E, \mathbf{r}),
\end{equation}
where $Y_{\ell, m}(\mathbf{r})$ are spherical harmonic functions, $c_{\ell, m}$ are coefficients and $R_\ell(E, \mathbf{r})$ is the solution of the radial wave equation:
\begin{equation}
    \frac{d^2(rR_\ell)}{dr^2} = \bigg[V(r) + \frac{\ell(\ell+1)}{r^2} - E\bigg](rR_\ell).
\end{equation}

\subsubsection{Boundary Conditions}

\begin{equation}
    \psi(\mathbf{k,R_2}) = e^{i\mathbf{k}\cdot\mathbf{t_{12}}}\psi(\mathbf{k,R_1})
\end{equation}
\begin{equation}
    \mathbf{n} \cdot \nabla\psi(\mathbf{k,R_2}) = e^{i\mathbf{k}\cdot\mathbf{t_{12}}}\mathbf{n} \cdot \psi(\mathbf{k,R_1})
\end{equation}

\begin{figure}[h!]
\centering 
\includegraphics[width=40mm]{images/single_cell_formulation.png}
\end{figure}

\subsection{Dynamical Aspects of Electrons}

\subsubsection{Free Electron Eigenfunction}
\begin{equation}
    W(k,x) = \frac{1}{\sqrt{L}}e^{ikx}
\end{equation}
These plane waves are eigenfunctions of the momentum operator, but Bloch-type wave functions are not.

\subsubsection{Free Electron Eigenvalue}
\begin{equation}
    E(k) = \frac{\hbar^2 k^2}{2m}
\end{equation}

\subsubsection{Velocity of Electrons in a Band}
\begin{equation}
    v(k) = \frac{1}{\hbar}\frac{dE(k)}{dk}
\end{equation}

\subsubsection{Action of an External Electric Field}
\begin{equation}
    k(t) = -\frac{eFt}{\hbar} + k_0
\end{equation}

\subsubsection{Effective Mass}
\begin{equation}
    m^* = \hbar^2\frac{d^2k}{dE(k)^2}
\end{equation}

\section{Electrons in Mesoscopic Systems}

\textit{From lecture slides}: ``the bridge between the macroscopic world of bulk materials and the microscopic world of atoms and molecules''.

\subsection{Relevant Length Scales}

\subsubsection{Fermi Energy}

\begin{equation}
    E_F = \frac{\hbar^2k_F^2}{2m}
\end{equation}

\subsubsection{Fermi Wavelength}

\begin{equation}
    \lambda_F = \frac{2\pi}{k_F} = \frac{2\pi\hbar}{\sqrt{2mE_F}}
\end{equation}

\subsubsection{Elastic Mean Free Path}

\begin{equation}
    \tau_e = \frac{l_e}{v_F}
\end{equation}

\subsubsection{Inelastic Mean Free Path}

\begin{equation}
    \tau_i = \frac{l_i}{v_F}
\end{equation}

\subsubsection{Cyclotron Frequency}

\begin{equation}
    \omega_c = \frac{v_F}{R_c} = \frac{eB}{m^*}
\end{equation}

\subsubsection{Cyclotron Frequency (Gaussian Units)}

\begin{equation}
    \omega_c = \frac{eB}{mc}
\end{equation}

\subsubsection{Centrifugal Force}

\begin{equation}
    F_c = \frac{m^* v_F^2}{R_c}
\end{equation}

\subsection{Relevant Energy Scales}
\subsubsection{Single-Electron Charging Energy}

\begin{equation}
    E_c = \frac{e^2}{2C}
\end{equation}

\subsection{Transverse Modes in 1D Conductor}

\begin{figure}[h!]
\centering 
\includegraphics[width=70mm]{images/1D_conductor.png}
\end{figure}

\subsubsection{Schrödinger Equation}

\begin{equation}
    \bigg(E_s + \frac{(i\hbar\nabla + e\mathbf{A})^2}{2m} + U(y)\bigg)\psi(x,y) = E\psi(x,y),
\end{equation}
where $E_s$ is the bottom of the band.

\subsubsection{Parabolic Band Potential}

\begin{equation}
    U(y) = \frac{1}{2}m\omega_0^2y^2
\end{equation}
This is achieved experimentally by using the electrostatic depletion of charges underneath split gates defined on a 2D electron gas (Prof. B.J. van Wees).

\subsubsection{First Landau Gauge}

\begin{equation}
    \mathbf{A} = (-By,0,0)
\end{equation}

\subsubsection{Quantisation of Modes}

Different from a linear Ohm's law and is independent of the material used.
\begin{figure}[h!]
\centering 
\includegraphics[width=70mm]{images/quantisation_modes.png}
\end{figure}

\noindent
To explain this we consider confined electrons $(U \neq 0)$ in a zero magnetic field $(B=0)$.

\subsubsection{Schrödinger Equation}

\begin{equation}
    \bigg(E_s + \frac{\hbar^2k^2}{2m} + \frac{p_y^2}{2m} + \frac{1}{2}m\omega_0^2y^2\bigg)\chi(y) = E\chi(y)
\end{equation}
\begin{equation}
    \chi_{n,k}(y) = u_n(q) = H_n(q)\exp{\bigg(-\frac{q^2}{2}\bigg)},
\end{equation}
where $H_n(q)$ are the Hermite polynomials.
\begin{equation}
    E(n,k) = E_s + \frac{\hbar^2k^2}{2m} + \bigg(n+\frac{1}{2}\bigg)\hbar\omega_0,
\end{equation}
where $\omega_0$ is the strength of the confining potential. Spacing between subbands is equal to $\hbar\omega_0$ and tighter confinement leads to the subbands being further apart.

\subsubsection{Velocity of Electrons in Transverse Mode}

\begin{equation}
    \mathbf{v}(\mathbf{k}) = \frac{1}{\hbar}\frac{dE(\mathbf{k})}{d\mathbf{k}}
\end{equation}

\subsubsection{Conductance}

Conductance of a ballistic conductor is given by:
\begin{equation}
    G = \sigma\frac{A}{L} = \lim_{L\to 0}\bigg(\sigma\frac{W}{L}\bigg) \to \infty.
\end{equation}
In reality, $G$ approaches a finite value $G_c$ when the length of the sample becomes smaller than then mean free path $(L << \ell)$. So, a ballistic conductor should have $0$ resistance due to a lack of scattering but it does, so where does this resistance come from? 

\subsubsection{Landauer Formula}

Contact conductance is given by:
\begin{equation}
    G_c = \frac{2e^2}{h}M
\end{equation}

\subsubsection{Conducting Channel Modes}

Number of modes in a conducting channel of width $W$:
\begin{equation}
    M = \text{Int}\bigg[\frac{k_F W}{\pi}\bigg] = \text{Int}\bigg[\frac{W}{(\lambda_F/2)}\bigg]
\end{equation}

\section{Electron Gas in Magnetic Fields}

\subsubsection{Landauer's Formula}

Number of modes in a conducting channel of width $W$:
\begin{equation}
    G = \frac{g_S e^2}{h}MT,
\end{equation}
where $T$ is the transmission probability of the wire.

\subsection{2D Free Electron Gas in a Magnetic Field}

Uniform magnetic field parallel to $z$:

\subsubsection{Effective Hamiltonian}

\begin{equation}
    H_0 = \frac{1}{2m}(p_x + eA_x)^2 + \frac{1}{2m}(p_y + eA_y)^2
\end{equation}
In the first Landau gauge, this reduces to:
\begin{equation}
    E_{nk_x} = \bigg(n+\frac{1}{2}\bigg)\hbar\omega_c,
\end{equation}
where $n$ represents the Landau level. So, the presence of a magnetic field modifies the continuous kinetic energy of a free electron gas into discrete energy levels. Eigenvalues are not gauge-dependent, but the eigenfunctions are.

\subsubsection{Eigenfunctions}

\begin{equation}
    \phi_{nk_x}(\mathbf{r}) = \frac{1}{\sqrt{L_x}}H_n(y-y_0)e^{ik_x x}
\end{equation}

\subsubsection{Wavefunction}

\begin{equation}
    \ell = \sqrt{2n+1}\ell_0 = \sqrt{2n+1}\sqrt{\frac{\hbar}{eB}}
\end{equation}

\subsubsection{Landau-Level Degeneracy}

\begin{equation}
    0 \leq y_0 = \frac{\hbar k_x}{eB} < L_y
\end{equation}
\begin{equation}
    k_x(\text{max}) = \frac{eBL_y}{\hbar}
\end{equation}
\begin{equation}
    N_L(H) = g_s\frac{L_x}{2\pi}\frac{eB}{\hbar}L_y = g_s\frac{eB}{2\pi\hbar}L_xL_y,
\end{equation}
where $g_s$ is the spin degeneracy.

\subsubsection{DOS at Zero Magnetic Field}

\begin{equation}
    D(E, H=0) = g_s\frac{m}{2\pi\hbar^2}L_xL_y\theta(E-E_s)
\end{equation}
\begin{equation}
    N(E, H=0) = D(E, H=0)\hbar\omega_c
\end{equation}

\subsubsection{Landau Level DOS}

\small
\begin{equation}
    D(E, H\neq0) = g_s\frac{eB}{2\pi\hbar}\sum_{n=0}^\infty \delta\bigg[E-E_s-\bigg(n+\frac{1}{2}\bigg)\hbar\omega_c\bigg]
\end{equation}
\normalsize

\subsubsection{Effect of Spin}

Including the spin leads to Zeeman splitting:
\begin{equation}
    E_n(B) = \bigg(n+\frac{1}{2}\bigg)\hbar\omega_c \pm \frac{1}{2}g\mu_B B
\end{equation}

\subsection{Quantum Hall Effect in 2D Electron Gas}

\subsubsection{Edge States}

The edge of the sample is modelled by a potential which rises steeply. Due to $E(x) = -\nabla U(x)$, charges at the edge of the channel acquire a drift velocity and the direction is opposite on either side of the sample so that in the absence of an electric field, the net current vanishes. There is also complete suppression of backscattering. Properties of edge states are:
\begin{itemize}
    \item{there are as many edge states as there are filled Landau levels in the bulk,}
    \item{each edge state is characterised by the quantum numbers $\{n,ky\}$,}
    \item{they are spatially separated by distances governed by the confining potential gradient,}
    \item{they have a width of typically the magnetic length, and}
    \item{they are one dimensional in nature.}
\end{itemize}

\begin{figure}[h!]
\centering 
\includegraphics[width=70mm]{images/quantum_hall.png}
\end{figure}

\subsubsection{Drift Velocity of Edge Charges}

\begin{equation}
    v_d = \pm\frac{|\mathbf{E}|}{|\mathbf{B}|}
\end{equation}

\subsubsection{Current Carried by Edge States}

Consider a system where the left reservoir emits electrons into the system, i.e. $\mu_1 > \mu_2$. Current is only transported by electrons in the range $\Delta\mu = \mu_1 - \mu_2$.

\begin{equation}
    J = g_s \frac{e^2}{h}V_\text{bias}
\end{equation}

\begin{figure}[h!]
\centering 
\includegraphics[width=70mm]{images/hall_current1.png}
\end{figure}
\begin{figure}[h!]
\centering 
\includegraphics[width=70mm]{images/hall_current2.png}
\end{figure}

\subsubsection{Total Conductance}

\begin{equation}
    G = N_{\text{max}}\frac{g_s e^2}{h}
\end{equation}

\section{Electron Gas in Magnetic Fields}

\subsection{Shubnikov-de Haas Oscillations}

Number of occupied Landau levels changes as the magnetic field is changed. $R_{xx}$ is a maximum when this number is a half-integer and the Fermi energy lies at the centre of the Landau level. These oscillations are periodic in inverse magnetic field.

\subsubsection{Temperature Condition}

\begin{equation}
    \hbar\omega_c > k_BT
\end{equation}

\subsubsection{Number of Occupied Landau Levels}

\begin{equation}
    N = \frac{n_s}{\big(\frac{2eB}{h}\big)}
\end{equation}

\subsubsection{Electron Density}

\begin{equation}
    n_s = \frac{2e}{h}\frac{1}{\big(\frac{1}{B_1} - \frac{1}{B_2}\big)}
\end{equation}

\subsubsection{Condition for the Formation of Landau Levels}

\begin{equation}
    \omega_c^{-1} << \tau
\end{equation}
\begin{equation}
    B >> \mu^{-1}
\end{equation}

\subsection{Systems with Multiple Bands}

\subsubsection{Conducitivty}

\begin{equation}
    \sigma_\text{total} = \sum_i \sigma_i
\end{equation}

\subsubsection{Resistivity}

\begin{equation}
    \rho_\text{tot} = \frac{1}{\sigma_\text{tot}}
\end{equation}

\subsection{Quantum Hall Plateaus and Disorder}

Disorder explains why experiment predicts flat plateaus but collisionless model predicts linear dependence of $R_{xy}$ on $B$. Continuous movement of the Fermi energy relies upon there being disorder in the system.

\subsection{Confined Electrons in Non-Zero Magnetic Field}

\subsubsection{Schrödinger Equation}

\small
\begin{equation}
    \bigg[E_s + \frac{(\hbar k + eBy)^2}{2m} + \frac{p_y^2}{2m} + \frac{1}{2}m\omega_0^2y^2\bigg]\chi(y) = E\chi(y)
\end{equation}
\normalsize

\subsubsection{Eigenvalues}

\small
\begin{equation}
    E(n,k) = E_s + \bigg(n + \frac{1}{2}\bigg)\hbar{\sqrt{\omega_0^2 + \omega_c^2}} + \frac{\hbar^2 k^2}{2m}\frac{\omega_0^2}{\omega_0^2 + \omega_c^2}
\end{equation}
\normalsize

\subsubsection{Wavefunction Centre}

\begin{equation}
    y_k = \frac{\hbar k}{eB} = v(n,k)\frac{\omega_0^2 + \omega_c^2}{\omega_0\omega_c^2}
\end{equation}

\subsubsection{Mass Change}

\begin{equation}
    m \to m\bigg(1 + \frac{\omega_0^2}{\omega_c^2}\bigg)
\end{equation}

\subsection{3D Free Electron Gas in Magnetic Field}

\subsubsection{Hamiltonian}

\begin{equation}
    H_0 = \frac{1}{2m}(\mathbf{p} + e\mathbf{A})^2
\end{equation}

\subsubsection{Gauge}

\begin{equation}
    \mathbf{A} = (-Hy,0,0)
\end{equation}

\subsubsection{Eigenvalues}

\begin{equation}
    E_{nk_z} = \bigg(n + \frac{1}{2}\bigg)\hbar\omega_c + \frac{\hbar^2 k_z^2}{2m},
\end{equation}
where each $n$ and $k_z$ state has degeneracy $eHS/h$ (excluding spin degeneracy).

\subsubsection{Density of States $(B=0)$}

\begin{equation}
    D(E,H=0) = 4\pi V(2m)^{3/2}h^{-3}\sqrt{E}
\end{equation}

\subsubsection{Density of States $(B\neq0)$}

\begin{equation}
\begin{split}
    D(E,H\neq0) &= \frac{1}{2}\hbar\omega_c A\sum_{n=0}^\infty\frac{1}{\sqrt{E - \big(n+\frac{1}{2}\big)\hbar\omega_c}} \\
    &\cdot \theta{\bigg[E - \bigg(n + \frac{1}{2}\bigg)\hbar\omega_c\bigg]} 
\end{split}
\end{equation}

\section{Superconductivity}

First arose from measurements of resistivity at low temperatures. In superconductors, the conductivity appears to become infinite below a certain temperature.

\subsection{Current in a Superconducting Ring}

\begin{equation}
    \frac{d\phi}{dt} = 0
\end{equation}

\subsection{Meissner-Oschenfeld Effect}

There exist zero-resistance states that are not superconductors. This effect shows that superconductors expel a weak magnetic field in thermal equilibrium.

\subsection{Perfect Diamagnetism}

Screening currents flowing around the edges of the sample maintain B = 0 inside the sample (Meissner-Ochsenfeld). These currents produce a magnetic field which cancels the applied external field.

\subsubsection{Internal Screening Currents}

\begin{equation}
    \mathbf{j}_\text{int} = \nabla \times \mathbf{M},
\end{equation}
where $\mathbf{M}$ is the magnetisation per unit volume.

\subsubsection{Magnetic Field}

\begin{equation}
    \mathbf{B} = \mu_0(\mathbf{H} + \mathbf{M})
\end{equation}

\subsubsection{Magnetic Susceptibility}

\begin{equation}
    \chi = \frac{d\mathbf{M}}{d\mathbf{H}}\bigg|_{\mathbf{H}=0}
\end{equation}
Imposing the Meissner constraint ($\mathbf{B}=0$), gives $\mathbf{M} = -\mathbf{H}$ and $\chi = -1$. Solids with $\chi < 0$ are diamagnets and superconductors are perfect diamagnets.

\subsection{Types of Superconductor}

\subsubsection{Type I}

The magnetic field remains zero inside the superconductor until suddenly the superconductivity is destroyed. The field where this happens is called the critical field, $H_c$.
\begin{figure}[h!]
\centering 
\includegraphics[width=70mm]{images/type1.png}
\end{figure}

\subsubsection{Type II}

There are two different critical fields, denoted $H_{c1}$ the lower critical field, and $H_{c2}$ the upper critical field. A type II superconductor will only keep out the whole magnetic field until $H_{c1}$ is reached, then vortices begin to form.

\begin{figure}[h!]
\centering 
\includegraphics[width=70mm]{images/type2.png}
\end{figure}

\subsection{Superconducting Phase Transition}

The superconducting state and the normal metallic state are separate thermodynamic phases of matter in just the same way as gas, liquid, and solid are different phases. Each such phase transition can be characterised by the nature of the singularities in specific heat and other thermodynamic variables at the transition, Tc.

\subsection{Ginzburg-Landau Theory}

\subsubsection{Order Parameter}

Phase transition is characterised by the order parameter (generally complex) which is zero in the disordered state above $T_c$ but non-zero below $T_c$.
\begin{equation}
    \psi = \begin{cases} 0\ \ \ \ T > T_c \\
    \psi(T) \neq 0\ \ \ \ T < T_c\end{cases}
\end{equation}

\subsubsection{Free Energy Density}

\begin{equation}
    f_s(T) = f_n(T) + a(T)|\psi|^2 + \frac{1}{2}b(T)|\psi|^4 + ...,
\end{equation}
where the $s$ subscript denotes the superconducting state and $n$ the normal state.
\begin{figure}[h!]
\centering 
\includegraphics[width=70mm]{images/gl1.png}
\end{figure}

\subsubsection{Order Parameter Near $T_c$}

\begin{equation}
    |\psi| = \begin{cases} \sqrt{\frac{\dot{a}}{b}}\sqrt{T - Tc}\ \ \ \ T < T_c \\
    0\ \ \ \ T > T_c
    \end{cases}
\end{equation}

\subsubsection{Set of Minima}

\begin{equation}
    \psi = |\psi|e^{i\theta}
\end{equation}

\subsubsection{Entropy per Unit Volume}

\begin{equation}
    S_s(T) - S_n(T) = -\frac{\dot{a}(T-T_c)}{b}
\end{equation}

\subsubsection{Heat Capacity per Unit Volume}

\begin{equation}
    C_{V_s} - C_{V_n} = \begin{cases} \frac{T\dot{a}^2}{b}\ \ \ \ T < T_c \\
    0\ \ \ \ T > T_c
    \end{cases}
\end{equation}

\subsubsection{Limitations of Mean Field Theories}

\begin{enumerate}
    \item{Mean-field theories fail to explain the critical region accurately because the assumption that all regions of the sample are the same then becomes particularly misplaced.}
    \item{Mean field theories ignore correlations and fluctuations which become very important near $T_c$. Very near to the critical temperature, large fluctuations are seen in the order parameter.}
\end{enumerate}

\subsection{Josephson Effect}

The Josephson effect is the phenomenon of supercurrent, a current that flows continuously without any voltage applied, across a device known as a Josephson junction (JJ).

\begin{figure}[h!]
\centering 
\includegraphics[width=70mm]{images/josephson.png}
\end{figure}

\subsubsection{Current Flow}

\begin{equation}
    I = I_c\sin{(\theta_2 - \theta_1)}
\end{equation}

\subsubsection{Order Parameter within Barrier}

\begin{figure}[h!]
\centering 
\includegraphics[width=70mm]{images/josephson_2.png}
\end{figure}
\begin{equation}
    \psi{(z)} = \psi_1e^{-\beta z} + \psi_2 e^{\beta(z - b)} 
\end{equation}

\subsection{The London Equation}

\begin{equation}
    \mathbf{j} = -\frac{n_s e^2}{m}\mathbf{A},
\end{equation}
where $\mathbf{j}$ is the current density, and $n_s$ is the density of superfluid electrons.

\subsection{Finite-Frequency Conductivity}

\begin{equation}
    \sigma(\omega) = \frac{\pi ne^2}{m}\delta(\omega) - \frac{ne^2}{i\omega m}
\end{equation}

\subsection{Rewritten London Equation}

\begin{equation}
    \nabla \times (\nabla \times \mathbf{B}) = -\frac{1}{\lambda^2}\mathbf{B}
\end{equation}

\subsection{Penetration Depth}

\begin{equation}
    \lambda = \sqrt{\frac{m}{\mu_0 ne^2}}
\end{equation}

\section{Superconductivity}

\subsection{Generalised London Equation}

\small
\begin{equation}
    \mathbf{j}(\mathbf{r}) = -\frac{ne^2}{m}\frac{3}{4\pi\zeta_0}\int\frac{\mathbf{R}(\mathbf{R}\cdot\mathbf{A}(r'))}{R^4}e^(-R/\zeta)d^3\mathbf{r'},
\end{equation}
\normalsize
where $\mathbf{R} = \mathbf{r-r'}$; $\zeta_0$ is the superconducting characteristic length or coherence length; and $1/\zeta = 1/\zeta_0 + 1/\ell$, where $\ell$ is the mean free path.

\subsection{Coherence Length}

\begin{equation}
    \zeta_0 = \frac{hv_F}{k_BT_c}
\end{equation}

\subsection{London Vortex}

Outside the core, apply the London equation. Inside the core, the magnetic field is a constant $B_0$.

\subsubsection{Outside Core for $\boldsymbol{\zeta_0} \mathbf{< r <<} \boldsymbol{\lambda}$}

\begin{equation}
    B_z(r) = \frac{\phi_0}{2\pi\lambda^2}\ln{\bigg(\frac{\lambda}{r}\bigg)},
\end{equation}
where $\phi_0 = h/2e$.

\subsubsection{Outside Core for $\mathbf{r \sim} \boldsymbol{\lambda}$}

\begin{equation}
    B_z(r) = \frac{\phi_0}{2\pi\lambda^2}\sqrt{\frac{\pi\lambda}{2r}}e^{-r/\lambda}
\end{equation}

\subsection{The Isotope Effect}

\begin{equation}
    T_c \sim M^{-\alpha},
\end{equation}
where $M$ is the ionic mass and $\alpha$ is the isotope coefficient, given by:
\begin{equation}
    \alpha = -\frac{M}{M-M^*}\frac{T_c - T_c^*}{T_c}.
\end{equation}

\subsection{BCS Theory of Superconductivity}

\subsubsection{Bare Coulomb Repulsion}

\begin{equation}
    V(\mathbf{r-r'}) = \frac{e^2}{4\pi\epsilon_0|\mathbf{r-r'}|}
\end{equation}

\subsubsection{Thomas-Fermi (Screening) Repulsion}

\begin{equation}
    V_\text{TF}(\mathbf{r-r'}) = \frac{e^2}{4\pi\epsilon_0|\mathbf{r-r'}|}
\end{equation}

\subsubsection{BCS Ground State}

\begin{equation}
    \chi(\mathbf{r-r'}) = \frac{1}{\mathcal{V}}\sum_\mathbf{k}\chi_\mathbf{k}e^{i\mathbf{k\cdot r}}e^{-i\mathbf{k\cdot r}}
\end{equation}
So, electrons with equal and opposite wavevectors are paired.

\subsubsection{Spread in Momentum}

The spread in momentum of the one-electron levels making up this pair is:
\begin{equation}
    \Delta \approx v_F\delta p
\end{equation}

\subsubsection{Spatial Range}

\begin{equation}
    \zeta_0 \sim \frac{1}{k_F}\frac{E_F}{\Delta}
\end{equation}

\subsubsection{Energy Gap}

For $T \approx T_c$:
\begin{equation}
    \Delta(T) = 1.74\Delta_0\bigg(1 - \frac{T}{T_c}\bigg)^{1/2}
\end{equation}
Electron tunnelling (Giaever tunnelling) across a junction formed by a normal metal-insulator-superconductor (NIS junction) can be used to confirm the presence of this energy gap. Another way to confirm this is by Andreev scattering, which also confirms the presence of Cooper pairs.

\subsubsection{Critical Field}

\begin{equation}
    \frac{H_c(T)}{H_c(0)} \approx \begin{cases} 1 - 1.06\bigg(\frac{T}{T_c}\bigg)^2\ \ \ \ T << T_c \\
    1.74\bigg(1 - \frac{T}{T_c}\bigg)\ \ \ \ T \approx T_c
    \end{cases}
\end{equation}

\section{Magnetism}

\subsection{Landau Theory of Ferromagentism}

Ferromagnetism refers to materials that can retain their magnetic properties when the magnetic field is removed.

\subsubsection{Order Parameter}

\begin{equation}
    F(M) = F_0 + a(T)M^2 + bM^4,
\end{equation}
where $a(T) = a_0(T - T_c)$.

\subsubsection{Minima}

\begin{equation}
    M = 0, \pm \bigg(\frac{a_0(T_c - T)^{1/2}}{2b}\bigg)
\end{equation}

\subsection{Microscopic Models of the Magnetic Interaction}

\subsubsection{Heisenberg Model}

\begin{equation}
    \hat{H} = -\sum_{\langle ij \rangle}J\mathbf{S_i}\cdot\mathbf{S_j}
\end{equation}

\subsubsection{Ising Model}

In the Ising model, the spins are restricted to up and down:
\begin{equation}
    \hat{H} = -\sum_{\langle ij \rangle}J S_i^z S_j^z
\end{equation}

\subsection{1D Ising Model}

\subsubsection{Ground State Hamiltonian}

\begin{equation}
    \hat{H} = -2J\sum_{i=1}^N S_i^z S_{i+1}^z
\end{equation}

\subsubsection{Ground State Energy}

\begin{equation}
    E = -\frac{NJ}{2}
\end{equation}

\subsubsection{Entropy Gain}

\begin{equation}
    S = k_B\ln N
\end{equation}
For an infinite chain, as long as $t\neq 0$, $S\to\infty$ and defects can spontaneously form, resulting in no long-range order.

\subsection{Consequences of Broken Symmetry}

\begin{itemize}
    \item{Phase transitions}
    \item{Rigidity}
    \item{Excitations}
    \item{Defects}
\end{itemize}

\subsection{Pauli Paramagnetism}

The Pauli principle means that each k-state in a metal can be doubly occupied. 

\subsubsection{Extra Electrons with Spin Up}

\begin{equation}
    n_\uparrow = \frac{1}{2}g(E_F)\mu_B B
\end{equation}

\subsubsection{Deficit of Electrons with Spin Down}

\begin{equation}
    n_\downarrow = \frac{1}{2}g(E_F)\mu_B B
\end{equation}

\subsubsection{Magnetisation}

\begin{equation}
    M = \mu_B(n_\uparrow - n_\downarrow) = g(E_F)\mu_B^2B
\end{equation}

\subsubsection{Magnetic Susceptibility}

\begin{equation}
    \chi = \frac{M}{H} \approx \frac{\mu_0 M}{B} = \mu_0\mu_B^2g(E_F) = \frac{3n\mu_0\mu_B^2}{2E_F}
\end{equation}

\subsection{Spontaneously Spin-Split Bands}

\subsubsection{Number of Electrons}
\begin{equation}
    N_+ = \int_0^{E_+}\rho(E)dE
\end{equation}
\begin{equation}
    N_- = \int_0^{E_-}\rho(E)dE
\end{equation}
At T=0:
\begin{equation}
    N_{tot} = \int_0^{E_F}\rho(E)dE
\end{equation}
\subsubsection{Number of Electrons Moved in Absence of Magnetic Field}

\begin{equation}
    n = \frac{g(E_F)\delta E}{2}
\end{equation}

\subsubsection{Kinetic Energy Change}

\begin{equation}
    \Delta E_k = \frac{1}{2}g(E_F)(\delta E)^2
\end{equation}
So, there is an energy cost to this process which makes it unfavourable.

\subsubsection{Potential Energy Change}

\begin{equation}
    \Delta E_p = -\frac{1}{2}Ug(E_F)(\delta E)^2,
\end{equation}
where $U = \mu_0\mu_B^2\lambda$ and $\lambda = H/M$.

\subsubsection{Total Energy Change}

\begin{equation}
    \Delta E = \frac{1}{2}g(E_F)(\delta E)^2(1 - Ug(E_F))
\end{equation}

\subsubsection{Total Magnetisation}
\begin{equation}
    M = \frac{(N_+ - N_-)}{V}\mu_B,
\end{equation}
where $V$ is the gas volume.

\subsubsection{Stoner Criterion}

For spontaneous magnetism, $\Delta E <0$.
\begin{equation}
    Ug(E_F) >> 1
\end{equation}

\subsubsection{Stoner Enhancement}

The susceptibility for metals close to the Stoner criterion is modified.
\begin{equation}
 \chi = \frac{\chi_P}{1-Ug(E_F)}
\end{equation}

\subsubsection{Majority and Minority Spin}

The up-spin subband is called the majority spin subband because it has more electrons and therefore determines the magnetization direction. The down-spin subband is called the minority spin subband.

\subsubsection{Polarisation}

\begin{equation}
    P = \frac{G_\uparrow - G_\downarrow}{G_\uparrow + G_\downarrow} = \frac{g_\uparrow(E_F) - g_\downarrow(E_F)}{g_\uparrow(E_F) + g_\downarrow(E_F)}
\end{equation}

\section*{SS1 - De Haas van Alphen Effect}
\textit{From Wikipedia:} magnetic susceptibility of a pure metal crystal oscillates as the intensity of the magnetic field B is increased. It can be used to determine the Fermi surface of a material. Oscillations of the magnetic moment of a metal as a function of the static magnetic field intensity are given by:
\begin{equation}
    \Delta\bigg(\frac{1}{B}\bigg) = \frac{2\pi e}{\hbar S},
\end{equation}
where $\Delta$ describes the period of oscillation and $S$ is the area of the external orbit of the Fermi surface in the direction of the applied field.

\section*{SS2 - Quantum Dots}

``Quantum dots (QDs) are man-made nanoscale crystals that that can transport electrons"

\subsection*{Constant Interaction Model}

\subsubsection*{Assumptions}

\begin{enumerate}
    \item{Coulomb interactions among electrons in the dot and between electrons in the dot and those in the environment, are parameterised by a single constant capacitance, C:
    $C = C_s + C_d + C_g$, where $s$ represents the source, $d$ represents the drain, and $g$ represents the gate.}
    \item{The discrete energy spectrum can be described independently of the number of electrons in the dot.}
\end{enumerate}
\section*{SS3 - Spin Injection and Accumulation in a Diffusive Conductor}
\subsection{Einstein Relation for Conductivity}
\begin{equation}
    \sigma = e^2DN(E_F),
\end{equation}
where $D$ is the diffusion constant and $N(E_F)$ is the density of states at the Fermi level.
\subsection{Net Spin Polarisation}
\begin{equation}
    P = \frac{N_\uparrow(E_F) - N_\downarrow(E_F)}{N_\uparrow(E_F) + N_\downarrow(E_F)},
\end{equation}
\subsection{Transmission Electron Spin Resonance}
Can a spin-polarised current enter a non-magnetic material and maintain its polarisation over a penetration depth of significant length?
\subsubsection{Magnetic Dipole Moment}
\begin{equation}
    \boldsymbol{\mu} = -\frac{ge}{mc}\mathbf{S} = \pm\frac{e}{2mc}\hbar = \pm\mu_B
\end{equation}
\subsubsection{Transition Energy}
Can be supplied by the absorption of photons, resulting ina non-equilibrium distribution of spin-polarised electrons near the Fermi level i.e. `spin accumulation'.
\begin{equation}
    \boldsymbol{\mu} = -\frac{ge}{mc}\mathbf{S} = \pm\frac{e}{2mc}\hbar = \pm\mu_B
\end{equation}
\section*{SS4 - Optical Processes and Excitons}

The dielectric function can tell us a lot about the electronic band structure of a crystal.

\subsection{Optical Reflectance}

\subsubsection{Reflectivity Coefficient}
\begin{equation}
    r(\omega) = \frac{E(\text{refl})}{E(\text{inc})} = \rho(\omega)e^{i\theta(\omega)}
\end{equation}
\subsubsection{Reflectivity Coefficient at Normal Incidence}
\begin{equation}
    r(\omega) = \frac{n(\omega) + iK(\omega) - 1}{n(\omega) + iK(\omega) + 1},
\end{equation}
where $n(\omega)$ is the refractive index, and $K(\omega)$ is the extinction coefficient.
\subsubsection{Dielectric Function}
\begin{equation}
    \sqrt{\epsilon(\omega)} = N(\omega) = n(\omega) + iK(\omega),
\end{equation}
where $N(\omega)$ is the complex refractive index.
\subsubsection{Reflectance}
\begin{equation}
    R = \frac{E^*(\text{refl})E(\text{refl})}{E^*(\text{inc})E(\text{inc})} = \frac{(n-1)^2+K^2}{(n+1)^2+K^2}
\end{equation}
\subsubsection{Kramers-Kronig Relations}
Allow us to find the real part of the response ($\alpha(\omega)$) of a linear passive system if we know the imaginary part of the response at all frequencies (and vice versa). \newline

\noindent
\textbf{First Relation}
\begin{equation}
    \alpha'(\omega) = \frac{2}{\pi}\mathds{P}\int_0^\infty \frac{s\alpha''(s)}{s^2-\omega^2}ds
\end{equation}
\noindent
\textbf{Second Relation}
\begin{equation}
    \alpha''(\omega) = -\frac{2\omega}{\pi}\mathds{P}\int_0^\infty \frac{\alpha'(s)}{s^2-\omega^2}ds
\end{equation}
\subsubsection{Electronic Interband Transitions}
\begin{equation}
    \hbar\omega = \epsilon_c(\mathbf{k}) - \epsilon_v(\mathbf{k})
\end{equation}
\subsection{Excitons}
An exciton is a bound electron-hole pair. The absorption of a photon by such a pair is responsible for the lack of crystal transparency near the energy gap.
\subsubsection{Frenkel Excitations}
An excited state of a single atom, but the excitation can hop from one atom to another by coupling between neighbours. Occur in materials with relatively small dielectric constants. 
\subsubsection{Mott-Wannier Excitations}
Occur in materials with relatively large dielectric constants e.g. semiconductors. Electrical screening tends to reduce the potential and radius of excitations is generally larger than the lattice spacing.
\subsubsection{EHD}
Stands for `excitation condensation into electron-hole drops' and describes a condensed phase of an electron-hole plasma when maintained at low temperature and irradiated by light.

\subsection{Raman Effect in Crystals}
\subsubsection{Selection Rules}
\begin{equation}
    \omega = \omega' \pm \Omega
\end{equation}
\begin{equation}
    \mathbf{k} = \mathbf{k}' \pm \mathbf{K},
\end{equation}
where $\omega$ and $\mathbf{k}$ refer to the incident photon; $\omega'$ and $\mathbf{k}'$ refer to the scattered photon; and $\Omega$ and $\mathbf{K}$ refer to the phonon.
\subsubsection{Energy Loss Function}
\begin{equation}
    \mathbf{E} = -\text{Im}\bigg[\frac{1}{\epsilon(\omega,\mathbf{k})}\bigg]
\end{equation}\section*{SS5 - Electron Tunnelling and Energy Bands}
\begin{figure}[h!]
\centering 
\includegraphics[width=70mm]{images/barrier.png}
\end{figure}
\subsection{Wavefunctions}
\subsubsection{Left}
\begin{equation}
    \psi_L(x) = A_L e^{iqx}+B_Le^{-iqx}
\end{equation}
\subsubsection{Barrier}
\begin{equation}
    \psi_I(x) = A_I e^{\beta x}+B_Ie^{-\beta x}
\end{equation}
\subsubsection{Right}
\begin{equation}
    \psi_R(x) = A_R e^{iqx}+B_Re^{-iqx},
\end{equation}
where:
\begin{equation}
    q^2(E) = \frac{2mE}{\hbar^2}
\end{equation}
\begin{equation}
    \beta(E) = \frac{2m(V_0 - E)}{\hbar^2}
\end{equation}
\subsection{Boundary Conditions}
\begin{equation}
    \psi_L(0) = \psi_I(0) \implies A_L + B_L = A_I + B_I
\end{equation}
\small
\begin{equation}
    \psi_L'(0) = \psi_I'(0) \implies A_Liq - B_Liq = A_I\beta - B_I\beta
\end{equation}
\normalsize
\subsection{Forward Transfer Matrix}
\subsubsection{General Expression}
\begin{equation}
\begin{split}
    \begin{pmatrix}A_R\\B_R\end{pmatrix} &= M(E)\begin{pmatrix}A_L\\B_L\end{pmatrix}\\
    &= \begin{pmatrix}
    m_{11}(E)&m_{12}(E)\\
    m_{21}(E)&m_{22}(E)
    \end{pmatrix}\begin{pmatrix}A_L\\B_L\end{pmatrix}
\end{split}
\end{equation}
\subsubsection{Barrier Expression $\mathbf{L}\to\mathbf{I}$}
\begin{equation}
    \begin{pmatrix}A_I\\B_I\end{pmatrix} = \frac{1}{2\beta}\begin{pmatrix}
    iq+\beta & -iq+\beta\\
    -iq+\beta & iq+\beta
    \end{pmatrix}\begin{pmatrix}A_L\\B_L\end{pmatrix}
\end{equation}
\subsubsection{Barrier Expression $\mathbf{I}\to\mathbf{R}$}
\small
\begin{equation}
    \begin{pmatrix}A_R\\B_R\end{pmatrix} = \frac{1}{2iq}\begin{pmatrix}
    (iq+\beta)e^{(-iq+\beta)b} & (iq-\beta)e^{(-iq-\beta)b}\\
    (iq-\beta)e^{(iq+\beta)b} & (iq+\beta)e^{(iq-\beta)b}
    \end{pmatrix}\begin{pmatrix}A_I\\B_I\end{pmatrix}
\end{equation}
\normalsize
\subsubsection{Diagonal Transfer Matrix}
Satisfies $|m_{11}(E)|=1$ or $m_{12}(E)=0$ and represents resonance energies.
\subsection{Reflection}
\subsubsection{Amplitude}
\begin{equation}
    r = \frac{B_L}{A_L}
\end{equation}
\subsubsection{Coefficient}
\begin{equation}
    R = rr^* = |r|^2
\end{equation}
\subsection{Transmission}
\subsubsection{Amplitude}
\begin{equation}
    t = \frac{A_R}{A_L}
\end{equation}
\subsubsection{Coefficient}
\begin{equation}
    T = tt^* = |t|^2
\end{equation}
\subsection{Rewritten Transfer Matrix}
\begin{equation}
    M(E) = \begin{pmatrix}
    1/t^* & -r^*/t^*\\
    -r/t & 1/t
    \end{pmatrix}
\end{equation}
\subsection{Scattering Matrix}
More numerically stable
\subsubsection{General Expression}
\begin{equation}
\begin{split}
    \begin{pmatrix}B_L\\A_R\end{pmatrix} &= S(E)\begin{pmatrix}A_L\\B_R\end{pmatrix}\\
    &=\frac{1}{m_{22}}\begin{pmatrix}
    -m_{21} & 1\\|M| & m_{12}
    \end{pmatrix}
\end{split}
\end{equation}
\subsection{Multiple Barriers}
\begin{equation}
    M^\text{tot} = M_n...M_2M_1
\end{equation}
\subsection{Single Barrier Transmission Coefficient}
\subsubsection{$\mathbf{0\leq E\leq V_0}$}
\begin{equation}
    T(E) = \frac{1}{1+\frac{V_0^2}{4E(V_0 - E)}\sinh^2\sqrt{\frac{2m(V_0-E)b^2}{\hbar^2}}}
\end{equation}
\subsubsection{$\mathbf{E\geq V_0}$}
\begin{equation}
    T(E) = \frac{1}{1+\frac{V_0^2}{4E(E-V_0)}\sin^2\sqrt{\frac{2m(E-V_0)b^2}{\hbar^2}}}
\end{equation}

\clearpage
\appendix

\section{Maths Appendix}
\subsection{Matrices}

\noindent
\textbf{Transpose}
\begin{equation}
    A_{ij}^T = A_{ji}
\end{equation}

\noindent
\textbf{Trace}
\begin{equation}
    tr(\mathbf{A}) = \sum{A_{ii}}
\end{equation}

\noindent
\textbf{Adjoint Matrix} \newline

\noindent
A matrix of minors, $\alpha_{ij}$, with signs attached: \newline
\begin{equation}
    \text{adj}(A) = 
    \begin{pmatrix}
    + & - & + \\
    - & + & - \\
    + & - & + \\
    \end{pmatrix}
\end{equation}

\noindent
\textbf{Systems of Linear Equations} \newline

\noindent
Can be represented as:
\begin{equation}
    \textbf{AX} = \textbf{B},
\end{equation}

\noindent
and can be solved by two methods: matrix inversion and Cramer's rule. \newline

\noindent
\textbf{Matrix Inversion Method}
\begin{equation}
    \mathbf{A}^{-1} = \frac{\text{adj}(A)}{|A|}
\end{equation}

\noindent
\textbf{Cramer's Rule}
\begin{equation}
    x_i = \frac{|\mathbf{C}(i)|}{|\mathbf{A}|}
\end{equation}

\noindent
\textbf{Matrix Eigenvalue Equation}
\begin{equation}
    \mathbf{A}\mathbf{r} = \lambda \mathbf{r}
\end{equation}
\noindent
Eigenvalues can be found by taking the determinant:
\begin{equation}
    |\mathbf{A} - \lambda \mathbf{I}| = 0.
\end{equation}

\subsection{Trigonometry}

\noindent
\textbf{Double-Angle Formulae}
\begin{equation}
    sin(\theta\pm\phi)=\sin\theta\cos\phi \pm \sin\phi\cos\theta
\end{equation}
\begin{equation}
    \sin(2\theta)=2\sin\theta\cos\theta
\end{equation}
\begin{equation}
    \cos(\theta\pm\phi)=\cos\theta\cos\phi \mp \sin\theta\sin\phi
\end{equation}
\begin{equation}
    \cos(2\theta)=\cos^2\theta-\sin^2\theta
\end{equation}
\begin{equation}
    \tan(\theta\pm\phi)=\frac{\tan\theta \pm \tan\phi}{1\mp\tan\theta\tan\phi}
\end{equation}
\begin{equation}
    \tan 2\theta=\frac{2\tan\theta}{1-\tan^2\theta}
\end{equation}

\noindent
\textbf{Hyperbolic Identities}
\begin{equation}
    \sinh x =\frac{e^x-e^{-x}}{2}=i\sin x
\end{equation}
\begin{equation}
    \cosh x =\frac{e^x+e^{-x}}{2}=\cos ix
\end{equation}
\begin{equation}
    \tanh x = \frac{\sinh x}{\cosh x} = \frac{e^x-e^{-x}}{e^x+e^{-x}}
\end{equation}
\begin{equation}
    \cosh^2 x - \text{sinh}^2 x = 1
\end{equation}
\begin{equation}
    \tanh^2 x + \text{sech}^2 x = 1
\end{equation}

\subsection{Series and Expansions}

\noindent
\textbf{Fourier Series}
\small
\begin{equation}
\begin{split}
    f(x) &= \frac{a_0}{2}\sum_{m=1}^\infty\bigg[a_m\cos{\bigg(\frac{m\pi x}{L}\bigg)} + b_m\sin{\bigg(\frac{m\pi x}{L}\bigg)}\bigg] \\
    &= \sum_{n=-\infty}^{\infty}c_n \exp{\bigg(\frac{in\pi x}{L}\bigg)}
\end{split}
\end{equation}
\normalsize

\noindent
where: 
\begin{equation}
    a_0 = \frac{1}{L}\int_{-L}^Lf(x)\ dx
\end{equation}
\begin{equation}
    a_n = \frac{1}{L}\int_{-L}^Lf(x)\cos{\bigg(\frac{n\pi x}{L}\bigg)}\ dx
\end{equation}
\begin{equation}
    b_n = \frac{1}{L}\int_{-L}^Lf(x)\sin{\bigg(\frac{n\pi x}{L}\bigg)}\ dx
\end{equation}
\begin{equation}
    c_n = \frac{1}{2L}\int_{-L}^Lf(x)\exp{\bigg(-\frac{in\pi x}{L}\bigg)}\ dx
\end{equation} \newline
\noindent
\textbf{Taylor Series}
\begin{equation}
    f(x) = \sum_{n=0}^\infty \frac{f^n(a)}{n!}(x-a)^n
\end{equation}

\noindent
\textbf{Maclaurin Series} \newline
\noindent
A Taylor series centred at zero.

\begin{equation}
    f(x) = \sum_{n=0}^\infty \frac{f^n(0)}{n!}(x)^n
\end{equation}

\noindent
\textbf{Binomial Expansion}
\small
\begin{equation}
    (a+b)^n &= a^n + \begin{pmatrix}n\\1\end{pmatrix}a^{n-1}b + ... +  \begin{pmatrix}n\\r\end{pmatrix}a^{n-r}b^r + ... + b^n 
\end{equation}
\normalsize

\noindent
where:
\begin{equation}
    \begin{pmatrix}n\\r\end{pmatrix} = ^nC_r = \frac{n!}{r!(n-r)!}
\end{equation}

\subsection{Transforms}

\noindent
\textbf{Fourier Transform}
\begin{equation}
    \mathcal{F}[f(x)] = \Tilde{f}(k) = \frac{1}{\sqrt{2\pi}}\int_{-\infty}^\infty f(x)e^{-ikx}\ dx
\end{equation}

\noindent
\textbf{Inverse Fourier Transform}
\begin{equation}
    \mathcal{F}^{-1}[\Tilde{f}(k)] = f(x) = \frac{1}{\sqrt{2\pi}}\int_{-\infty}^\infty \Tilde{f}(k)e^{ikx}\ dk
\end{equation}

\noindent
\textbf{Fourier Transform Properties}
\begin{itemize}
    \item{$\mathcal{F}[af(x) + bg(x)] = a\Tilde{f}(k) + b\Tilde{g}(k)$}
    \item{$\mathcal{F}[f(ax)] = \frac{1}{|a|}\Tilde{f}\big(\frac{k}{a}\big)$}
    \item{$\mathcal{F}[f'(x)] = ik\Tilde{f}(k)$}
    \item{$\mathcal{F}[f(x-x_0)] = e^{-ikx_0}\Tilde{f}(k)$}
    \item{$\int_{-\infty}^\infty |f(x)|^2\ dx = \int_{-\infty}^\infty |\Tilde{f}(k)|^2\ dk$}
\end{itemize}

\noindent
\textbf{Laplace Transform}
\begin{equation}
    \Tilde{f}(s) = \int_0^\infty f(t)e^{-st}\ dt
\end{equation}

\noindent
\textbf{Laplace Transform Properties}
\begin{itemize}
    \item{$L[f^{'}(t)] = -f(0) + sL[f(t)]$} 
    \item{$L[f^{n}(t)] = -f(0) + s^nL[f(t)] - s^{n-1}f(0) - s^{n-2}\dot{f}(0) - ... - \frac{d^nf(0)}{dt^n}$} 
\end{itemize}

\subsection{Calculus}

\noindent
\textbf{L'Hôpital's Rule}
\begin{equation}
    \lim_{x \to a}\bigg(\frac{f(x)}{g(x)}\bigg) = \lim_{x \to a}\bigg(\frac{f'(x)}{g'(x)}\bigg)
\end{equation}
\noindent
This is useful when $f(a) = g(a) = 0$ but $f'(a) \neq 0$ and $g'(a) \neq 0$. \newline

\noindent
\textbf{Chain Rule}
\begin{equation}
    \frac{df(u(x))}{dx} = \frac{du}{dx}\frac{df}{du}
\end{equation}

\noindent
\textbf{Product Rule}
\begin{equation}
    \frac{df(u(x)v(x))}{dx} = u\frac{dv}{dx} + v\frac{du}{dx}
\end{equation}

\noindent
\textbf{Quotient Rule}
\begin{equation}
    \frac{df\big(\frac{u(x)}{v(x)}\big)}{dx} = \frac{v\frac{du}{dx} - u\frac{dv}{dx}}{v^2}
\end{equation}

\noindent
\textbf{Partial Differentiation}
\begin{equation}
    df(x,y) = \frac{\partial f}{\partial x}dx + \frac{\partial f}{\partial y}dy
\end{equation}

\noindent
\textbf{Reciprocal Theorem}
\begin{equation}
    \frac{\partial x}{\partial z} = \frac{1}{\frac{\partial z}{\partial x}}
\end{equation}

\noindent
\textbf{Reciprocity Theorem}
\begin{equation}
    \frac{\partial x}{\partial y}\frac{\partial y}{\partial z}\frac{\partial z}{\partial x} = -1
\end{equation}

\noindent
\textbf{Jacobian}
\footnotesize
\begin{equation}
    \iiint_D{f(x,y,z)}\ dxdydz \to \iiint_D{f(u,v,w)}\ |\mathbf{J}|dudvdw
\end{equation}
\normalsize
\begin{equation}
    \mathbf{J} =
    \begin{vmatrix}
    \frac{\partial x}{\partial u} & \frac{\partial x}{\partial v} & \frac{\partial x}{\partial w} \\
    \frac{\partial y}{\partial u} & \frac{\partial y}{\partial v} & \frac{\partial y}{\partial w} \\
    \frac{\partial z}{\partial u} & \frac{\partial z}{\partial v} & \frac{\partial z}{\partial w} \\
    \end{vmatrix}
\end{equation}

\noindent
\textbf{Curve Length}
\begin{equation}
\begin{split}
    C &= \int_a^b\sqrt{1 + \bigg(\frac{dy}{dx}\bigg)^2}\ dx \\
    &= \int_a^b\sqrt{\bigg(\frac{dx}{dt}\bigg)^2 + \bigg(\frac{dy}{dt}\bigg)^2}\ dt 
\end{split}
\end{equation}

\noindent
\textbf{Area Under Curves}
\begin{equation}
\begin{split}
    A &= \int_{x_1}^{x_2} F[x, y(x)]\sqrt{1 + \bigg(\frac{dy}{dx}\bigg)^2}\ dx \\
    &= \int_{y_1}^{y+2} F[x(y), y]\sqrt{\bigg(\frac{dx}{dy}\bigg)^2 + 1}\ dy \\
    &= \int_{t=a}^{t=b} f[x(t), y(t)]\sqrt{\bigg(\frac{dx}{dt}\bigg)^2 + \bigg(\frac{dy}{dt}\bigg)^2}\ dt
\end{split}
\end{equation}

\noindent
\textbf{Green's Theorem}
\begin{equation}
    \oint_c \big(Pdx + Qdy\big) = \iint \bigg(\frac{\partial Q}{\partial x} - \frac{\partial P}{\partial y}\bigg)\ dA
\end{equation}

\noindent
\textbf{Area of Surface}
\begin{equation}
    A = \iint_S \sqrt{1 + \bigg(\frac{dz}{dx}\bigg)^2 + \bigg(\frac{dz}{dy}\bigg)^2}\ dxdy    
\end{equation}

\noindent
\textbf{Dirac Delta Function}
\begin{equation}
    f(X) = \int_\infty^\infty f(x)\delta(x-X)dx
\end{equation}

\noindent
\textbf{Properties of the Dirac Delta Function}
\begin{itemize}
    \item{Symmetry: $\delta(-x) = \delta(x)$}
    \item{$\delta[a(x-X)] = \frac{1}{|a|}\delta(x-X)$}
    \item{$\int_\infty^\infty\delta'(x)f(x)dx = -f'(0)$}
    \item{$\frac{d}{dx}\delta(x) = -\frac{1}{x}\delta(x)$}
\end{itemize}

\noindent
\textbf{Trigonometric Functions}

\renewcommand{\arraystretch}{2}
\begin{center}
\begin{supertabular}{|P{3cm}|P{3cm}|}
    \hline
    \textbf{Function} & \textbf{Derivative} \\ \hline
    $\sin(x)$ & $\cos(x)$ \\ \hline
    $\cos(x)$ & $-\sin(x)$ \\ \hline
    $\tan(x)$ & $\sec^2(x)$ \\ \hline
    $\csc(x)$ & $-\cot(x)\csc(x)$ \\ \hline
    $\sec(x)$ & $\sec(x)\tan(x)$ \\ \hline
    $\cot(x)$ & $-\csc^2(x)$ \\ \hline
    $\arcsin(x)$ & $\frac{1}{\sqrt{1-x^2}}$ \\ \hline
    $\arccos(x)$ & $-\frac{1}{\sqrt{1-x^2}}$ \\ \hline
    $\arctan(x)$ & $\frac{1}{\sqrt{1+x^2}}$ \\ \hline
    $\arccsc(x)$ & $-\frac{1}{|x|\sqrt{x^2-1}}$ \\ \hline
    $\arcsec(x)$ & $\frac{1}{|x|\sqrt{x^2-1}}$ \\ \hline
    $\arccot(x)$ & $-\frac{1}{\sqrt{1+x^2}}$ \\ \hline
\end{supertabular} \newline
\end{center}

\noindent
\textbf{Hyperbolic Functions}

\renewcommand{\arraystretch}{2}
\begin{center}
\begin{supertabular}{|P{3cm}|P{3cm}|}
    \hline
    \textbf{Function} & \textbf{Derivative} \\ \hline
    $\sinh(x)$ & $\cosh(x)$ \\ \hline
    $\cosh(x)$ & $\sinh(x)$ \\ \hline
    $\tanh(x)$ & $\sech^2(x)$ \\ \hline
    $\csch(x)$ & $-\coth(x)\csch(x)$ \\ \hline
    $\sech(x)$ & $-\sech(x)\tanh(x)$ \\ \hline
    $\coth(x)$ & $-\csch^2(x)$ \\ \hline
    $\arcsinh(x)$ & $\frac{1}{\sqrt{x^2+1}}$ \\ \hline
    $\arccosh(x)$ & $\frac{1}{\sqrt{x^2-1}}$ \\ \hline
    $\arctanh(x)$ & $\frac{1}{\sqrt{1-x^2}}$ \\ \hline
    $\arccsch(x)$ & $-\frac{1}{|x|\sqrt{1+x^2}}$ \\ \hline
    $\arcsech(x)$ & $-\frac{1}{x\sqrt{1-x^2}}$ \\ \hline
    $\arccoth(x)$ & $\frac{1}{\sqrt{1-x^2}}$ \\ \hline
\end{supertabular}
\end{center}
\renewcommand{\arraystretch}{2}

\subsection{Vector Calculus}

\noindent
\textbf{Directional Gradient}
\begin{equation}
    \text{Directional Gradient} = \frac{\nabla\phi \cdot \mathbf{a}}{|\mathbf{a}|}
\end{equation}

\noindent
\textbf{Divergence Theorem}
\begin{equation}
    \iiint_V{(\nabla \cdot \textbf{A})dV} = \oiint_S{(\textbf{A} \cdot \textbf{n}) dS}
\end{equation}

\noindent
\textbf{Stoke's Theorem}
\begin{equation}
    \oint_l{\textbf{A} \cdot d\textbf{l}} = \iint_S{(\nabla \times \textbf{A}) \cdot d\textbf{S}}
\end{equation}

\noindent
\textbf{Convolution Theorem}
\begin{equation}
    \mathcal{F}(f \otimes g) = \sqrt{2\pi}\mathcal{F}(f)\mathcal{F}(g)
\end{equation}

\subsection{Ordinary Differential Equations}

\noindent
\textbf{Separate Variables}
\begin{equation}
    \frac{dy}{dx} = g(x)h(y)
\end{equation}
\begin{equation}
    \int\frac{dy}{h(y)} = \int g(x)\ dx
\end{equation}

\noindent
\textbf{Homogeneous}
\begin{equation}
    \frac{dy}{dx} = f\bigg(\frac{y}{x}\bigg) = f(v)
\end{equation}
\begin{equation}
    \frac{dy}{dx} = x\frac{dv}{dx} + v = f(v)
\end{equation}

\noindent
\textbf{Exact Equation}
\begin{equation}
    P(x, y)dx + Q(x, y)dy = 0,
\end{equation}
\noindent
where:
\begin{equation}
    P(x, y) = \frac{\partial F}{\partial x}\bigg|_y\ \ \text{and}\ \ Q(x, y) = \frac{\partial F}{\partial y}\bigg|_x.
\end{equation}
\noindent
Equation is exact if:
\begin{equation}
    \frac{\partial P}{\partial y} = \frac{\partial Q}{\partial x}.
\end{equation}
\noindent 
Solve for $F$ by integrating $P$ and $Q$ and define constants so that the equations match. Rearrange to get $y$ in terms of $x$. \newline

\noindent
\textbf{Particular Integrals}
\begin{equation}
    \frac{dy}{dx} + P(x)y = Q(x)
\end{equation}
\begin{equation}
    I = e^{\int{P(x)dx}} 
\end{equation}
\begin{equation}
    \frac{d}{dx}(Iy) = IQ
\end{equation}
\noindent
Solve for y. \newline

\begin{sidewaystable*}
\subsection{Coordinate Systems}
\centering
\def\arraystretch{1.5}
\resizebox{\textwidth}{!}{\begin{tabular}{|c|c|c|c|}

\hline%------------------------------------------------------------------------
\textbf{Operation} & \textbf{Cartesian $(x,y,z)$}	& \textbf{Cylindrical $(\rho,\phi,z)$} &	\textbf{Spherical $(r,\theta,\phi)$}
\\
\hline%------------------------------------------------------------------------
\multirow{\textbf{Definition}} & $\displaystyle x=x$
 & $\displaystyle x=\rho\cos\phi $ & $\displaystyle x=r\sin\theta\cos\phi $\\
 &$\displaystyle y=y$ & $\displaystyle y=\rho\sin\phi$  & $x=r\sin\theta\cos\phi $\\
& $\displaystyle z=z$ & $\displaystyle z=z$ & $\displaystyle z=r\cos\theta$\\ 
\hline %------------------------------------------------------------------------
&&&\\[-0.5cm]

\multirow{\textbf{Unit Vectors}} & $\displaystyle \hat{\boldsymbol{\rho}}=\frac{x\hat{\mathbf{x}}+y\hat{\mathbf{y}}}{\sqrt{x^2+y^2}}$
 &$\hat{\mathbf x} =\cos\phi\hat{\boldsymbol{\rho}} - \sin\phi\boldsymbol{\hat{\phi}}$ & $\hat{\mathbf x} = \sin\theta\cos\phi\boldsymbol{\hat{r}} + \cos\theta\cos\phi\boldsymbol{\hat{\theta}}-\sin\phi\boldsymbol{\hat{\phi}}  $\\
 
 &$\displaystyle \hat{\boldsymbol{\phi}}=\frac{-y\hat{\mathbf{x}}+x\hat{\mathbf{y}}}{\sqrt{x^2+y^2}}$
  & $\hat{\mathbf y} = \sin\phi\boldsymbol{\hat{\rho}} + \cos\phi\boldsymbol{\hat{\phi}}$  & $\hat{\mathbf y} = \sin\theta\sin\phi\boldsymbol{\hat{r}} + \cos\theta\sin\phi\boldsymbol{\hat{\theta}}+\cos\phi\boldsymbol{\hat{\phi}} $\\
 
 &$\mathbf{\hat{r}}         = \displaystyle\frac{x \hat{\mathbf x} + y \hat{\mathbf y} + z \mathbf{\hat{z}}}{\sqrt{x^2+y^2+z^2}}$ & $\unit{z}=\unit{z}$ & $\displaystyle \boldsymbol{\hat{\theta}} = \frac{x z \hat{\mathbf x} + y z \hat{\mathbf y} - \left(x^2 + y^2\right) \mathbf{\hat{z}}}{\sqrt{x^2+y^2} \sqrt{x^2+y^2+z^2}} $
\\&&&
\\[-0.5cm]
\hline %------------------------------------------------------------------------
&&&\\[-0.5cm]
\textbf{Grad ($\nabla f$)}
 & $\displaystyle{\partial f \over \partial x}\hat{\mathbf x} + {\partial f \over \partial y}\hat{\mathbf y}
 + {\partial f \over \partial z}\mathbf{\hat{z}}$
  & $\displaystyle{\partial f \over \partial \rho}\boldsymbol{\hat{\rho}}
  + {1 \over \rho}{\partial f \over \partial \phi}\boldsymbol{\hat{\phi}}
  + {\partial f \over \partial z}\mathbf{\hat{z}}$
   & $\displaystyle{\partial f \over \partial r}\boldsymbol{\hat{r}}
   + {1 \over r}{\partial f \over \partial \theta}\boldsymbol{\hat{\theta}}
   + {1 \over r\sin\theta}{\partial f \over \partial \phi}\boldsymbol{\hat{\phi}}$
   \\[0.3cm]
\hline%------------------------------------------------------------------------
&&&\\[-0.5cm]
\textbf{Div ($\nabla \cdot \boldsymbol{a}$)} & $\displaystyle{\partial A_x \over \partial x} + {\partial A_y \over \partial y} + {\partial A_z \over \partial z}$ &$\displaystyle{1 \over \rho}{\partial \left( \rho A_\rho  \right) \over \partial \rho}
+ {1 \over \rho}{\partial A_\phi \over \partial \phi}
+ {\partial A_z \over \partial z}$& $\displaystyle{1 \over r^2}{\partial \left( r^2 A_r \right) \over \partial r}
+ {1 \over r\sin\theta}{\partial \over \partial \theta} \left(  A_\theta\sin\theta \right)
+ {1 \over r\sin\theta}{\partial A_\phi \over \partial \phi}$\\[0.3cm]
\hline%------------------------------------------------------------------------
& & &\\[-0.5cm]
\textbf{Curl ($\nabla \times \boldsymbol{a}$)} &
$\displaystyle \left | \begin{array}{c c c}
\boldsymbol{\hat{x}} & \boldsymbol{\hat{y}} & \boldsymbol{\hat{z}}\\
\displaystyle\pardif{}{x} & \displaystyle\pardif{}{y} & \displaystyle\pardif{}{z}\\
a_x & a_y & a_z\\
\end{array}\right|$ &
$\displaystyle \frac{1}{\rho}\left | \begin{array}{c c c}
\boldsymbol{\hat{\rho}} & \rho\boldsymbol{\hat{\phi}} & \boldsymbol{\hat{z}}\\
\displaystyle\pardif{}{\rho} & \displaystyle\pardif{}{\phi} & \displaystyle\pardif{}{z}\\
a_\rho & \rho a_\phi & a_z\\

\end{array}\right|$ & 
$\displaystyle \left | \begin{array}{c c c}
\displaystyle\frac{\boldsymbol{\hat{r}}}{r^2\sin\theta}
& \displaystyle\frac{\boldsymbol{\hat{\theta}}}{r\sin\theta} & \displaystyle\frac{\boldsymbol{\hat{\phi}}}{r}\\
\displaystyle\pardif{}{r} & \displaystyle\pardif{}{\theta} & \displaystyle\pardif{}{\phi}\\
a_r & r a_\theta & r\sin\theta a_\phi\\

\end{array}\right|$\\&&&
\\[-0.5cm]
\hline%------------------------------------------------------------------------
\textbf{Area Element (d$\boldsymbol{A}$)} & d$x$d$y\boldsymbol{\hat{z}}$ & $\rho$d$\rho$d$\phi\boldsymbol{\hat{z}}$ & $r^2\sin\theta$d$\theta$d$\phi\boldsymbol{\hat{r}}$\\
\hline%------------------------------------------------------------------------
\textbf{Volume Element (d$V$)} & d$x$d$y$d$z$ & $\rho$d$\rho$d$\phi$d$z$ & $r^2\sin\theta$d$\theta$d$\phi$d$r$\\
\hline%------------------------------------------------------------------------
\end{tabular}}
\end{sidewaystable*}

\end{document}
